%% The MIT License (MIT)
%%
%% Copyright (c) 2015 Daniil Belyakov
%%
%% Permission is hereby granted, free of charge, to any person obtaining a copy
%% of this software and associated documentation files (the "Software"), to deal
%% in the Software without restriction, including without limitation the rights
%% to use, copy, modify, merge, publish, distribute, sublicense, and/or sell
%% copies of the Software, and to permit persons to whom the Software is
%% furnished to do so, subject to the following conditions:
%%
%% The above copyright notice and this permission notice shall be included in all
%% copies or substantial portions of the Software.
%%
%% THE SOFTWARE IS PROVIDED "AS IS", WITHOUT WARRANTY OF ANY KIND, EXPRESS OR
%% IMPLIED, INCLUDING BUT NOT LIMITED TO THE WARRANTIES OF MERCHANTABILITY,
%% FITNESS FOR A PARTICULAR PURPOSE AND NONINFRINGEMENT. IN NO EVENT SHALL THE
%% AUTHORS OR COPYRIGHT HOLDERS BE LIABLE FOR ANY CLAIM, DAMAGES OR OTHER
%% LIABILITY, WHETHER IN AN ACTION OF CONTRACT, TORT OR OTHERWISE, ARISING FROM,
%% OUT OF OR IN CONNECTION WITH THE SOFTWARE OR THE USE OR OTHER DEALINGS IN THE
%% SOFTWARE.

% The font could be set to Windows-specific Calibri by using the 'calibri' option
\documentclass[]{mcdowellcv}
\usepackage{lmodern}
% For mathematical symbols
\usepackage{amsmath}
\usepackage{url}
\usepackage[svgnames]{xcolor}
\usepackage[colorlinks=true, linkcolor=Maroon, urlcolor=Maroon, citecolor=blue]{hyperref}
\usepackage[symbol]{footmisc}


\renewcommand{\thefootnote}{\fnsymbol{footnote}}


\usepackage[backend=biber,sorting=none]{biblatex}
\addbibresource{ref.bib}


% Set applicant's personal data for header
\name{Sanjeev Kumar}
\email{sanjeev90an@gmail.com}
\github{https://github.com/sanjeevk42}
\linkedin{https://linkedin.com/in/sanjeevk42}


\address{Phone: +49 15211918761 \linebreak Address: Felsennelkenanger 13 \linebreak Munich 80937}
\contacts{}


\begin{document}

    % Print the header
    \makeheader
    
    % academic projects
    \begin{cvsection}{Academic Projects}
    \begin{cvsubsection}[2]{Master Thesis}{CAMP\footnote{\href{http://campar.in.tum.de/Chair/ResearchIssueComputerVision}{http://campar.in.tum.de/Chair/ResearchIssueComputerVision}}, Technical University of Munich}{June 2017 -- Feb 2018}
    \textit{Multiple Action Prediction in Deep Reinforcement Learning (Python, Tensorflow, OpenAI Gym)}
        \begin{itemize}
            \item Proposed a new formulation for policy gradient reinforcement learning algorithms in continuous action space problems.
            \item The method predicts multiple action values at each state which facilitates better exploration during training and the agent converges to a better policy.
            \item Evaluated and compared the performance of the proposed formulation against other continuous control algorithms (A3C, DDPG, SVG(0)) on various Mujoco environments and TORCS car simulator for driving task.
        \end{itemize}
    \end{cvsubsection}
    
    \begin{cvsubsection}{Inter-Disciplinary Project}{Technical University of Munich}{Oct 2016 -- Mar 2017}
    \textit{Cell Detection in Lens-free Microscopy Videos (Python, Keras) \cite{rempfler-2017}}
        \begin{itemize}
        \item The aim of the project was to detect and localize cells in lens-free microscopy image sequences using deep convolutional neural networks (CNN).
        \item In this project, we experimented with different deep CNN architectures (FCN, UNet, DetectNet) and achieved best detection score of 95\% (F1) with fully convolutional ResNet-50.
        \end{itemize}
    \end{cvsubsection}
    
    \end{cvsection}
    
    % work experience
    \begin{cvsection}{Employment}

            \begin{cvsubsection}{Software Engineer}{Lyft Germany Gmbh}{Aug 2018 -- Present}    
        \textit{Detection of Road Surface for Automatic Map Building}
            \begin{itemize}
                \item Developing models for detection of road surface elements (lane boundaries, crosswalks etc.) for building the map.
                \item The map is used by prediction and planning for finding the car trajectory.
            \end{itemize}
        \end{cvsubsection}
    
            \begin{cvsubsection}{Student Tutor}{Technical University of Munich}{Oct 2017 -- March 2018}    
        \textit{Tracking and Detection in Computer Vision (IN2210) \footnote{\href{http://campar.in.tum.de/Chair/TeachingWs17TDCV}{http://campar.in.tum.de/Chair/TeachingWs17TDCV}}}
            \begin{itemize}
                \item The course is offered every winter semester to master students at TU Munich and is attended by more than 100 students. 
                \item Involved in creating assignments for the course and helped students with the homework. 
            \end{itemize}
        \end{cvsubsection}
    
        \begin{cvsubsection}{Work Student\footnote{Machine Learning and Computer Vision\label{ws}}}{Logivations Gmbh}{April 2017 - March 2018}
            \textit{Multi Label Classification Using Learning by Association (Python, Tensorflow)}
            \begin{itemize}
                \item Extended the semi-supervised image classification approach purposed by Haeusser at el. \cite{haeusser-cvpr-17} to multi-label classification.
                \item Achieved test error of 1.7\% on the artificially generated multi-label MNIST dataset using 10 supervised samples per class and 5000 total unsupervised samples. The test error without unsupervised samples was around 20\%.
            \end{itemize}
                
            \textit{Barcode Detector and Decoder (Python, C++, Caffe)}
            \begin{itemize}
                \item Developed an application which detected and decoded the barcodes of different items placed on a truck coming into the warehouse. The continuous video feed was captured by a PTZ camera and barcodes were decoded by zooming onto the specific location where barcode was detected.
                \item The detection of barcodes was done using RFCN.
            \end{itemize}
            
        \end{cvsubsection}
        
        \begin{cvsubsection}{Work Student\textsuperscript{\ref{ws}}}{Terraloupe Gmbh}{Aug 2016 - April 2017}
            \textit{Roof Area Estimation (Python, Keras, Shapely)}
            \begin{itemize}
                \item Experimented with various deep convolutional network architectures (PSPNet, UNet etc.) for semantic segmentation of roof objects (roof, chimney etc.) from aerial images.
                \item Implemented computational geometric algorithms (polygon merge, line merge etc.) to generate refined shape boundaries for detected objects. The roof area was calculated from the point cloud of given region and roof object boundaries.
            \end{itemize}
        \end{cvsubsection}

  \newpage        
          
        \begin{cvsubsection}{Software Development Engineer}{Amazon, India}{Oct 2014 -- Oct 2015}        
        \textit{Missed Call Based Expiry (Java, SQS, DynamoDB)}
            \begin{itemize}
                \item Developed an application which sends notifications (SMS, email) to customers for older ads live on junglee.com and customers can then delete their ads by giving a missed call on a toll-free number. 
                \item Using this mechanism, the Ad Defect Rate on junglee.com went down by 15\%.
            \end{itemize}
        \end{cvsubsection}

        \begin{cvsubsection}{Senior Software Developer}{Drishti Soft Solutions, India}{June 2012 -- Oct 2014}
            \textit{REST API for Ameyo (Java, Xtend, Jersey)}
            \begin{itemize}
                \item Designed REST-based APIs for Ameyo (Drishti's call center software suite). The project involved building a domain specific language (DSL) for API modeling and writing a code generator for generating application artifacts such as java glue code, documentation.
                \item This API provided a framework for third parties to integrate call center functionalities into their applications.
            \end{itemize}
            
            \textit{Stats Manager (Java, H2)}
            \begin{itemize}
                \item Designed and implemented a module for generating real-time call center statistics. 
                \item The module listened to call center application events and maintained an in-memory database of stats which could be consumed by API calls or by subscribing to notifications.
            \end{itemize}
        \end{cvsubsection}
    \end{cvsection}
    

    \begin{cvsection}{Education}
        \begin{cvsubsection}{Munich, Germany}{Technical University of Munich}{Oct 2015 -- March 2018}
            \begin{itemize}
                \item M.Sc. in Informatics, GPA: 1.7/1.0
                \item Graduate Coursework: Machine Learning; Variational Methods; Multiple View Geometry; Deep Learning for Computer Vision; Vision Based Navigation; Virtual Machines; Mining Massive Datasets.
            \end{itemize}
        \end{cvsubsection}
        \begin{cvsubsection}{Hamirpur, India}{National Institute of Technology}{July 2008 -- May 2012}
            \begin{itemize}
                \item B.Tech. in Computer Science and Engineering,  CGPA: 7.69/10.0
                \item Undergraduate Coursework: Object Oriented Programming; Data Structure and Algorithms; Theory of Computation; Operating System; Computer Networks; Computer Architecture.
            \end{itemize}
        \end{cvsubsection}
    \end{cvsection}
    
    \begin{cvsection}{Technical Skills}
        \begin{cvsubsection}{}{}{}    
            \begin{itemize}
                \item Programming Languages: Python (Advanced), C++ (Advanced), Java (Advanced), C (Intermediate), Matlab (Intermediate), JavaScript (Basic).
                \item ML/CV Toolkits: Tensorflow, Caffe, PyTorch, Keras, OpenCV, ROS, SciPy, Rasterio.
                \item Application Developement: REST, Apache Tomcat, Spring, Hibernate, Xtend, Eclipse Modeling Framework, PostgreSQL, Flask.
                \item Others: Git, GNU/Linux, LaTeX, Docker, Amazon Web Services.
            \end{itemize}
        \end{cvsubsection}
    \end{cvsection}
    
    \begin{cvsection}{Additional Experience and Awards}
        \begin{cvsubsection}{}{}{}    
            \begin{itemize}
                \item Participated in the annual hackathon at Amazon and won third prize in Junglee Hackathon’15.
                \item Awarded with Drishti Excellence Award in October 2013, for gaining domain knowledge quickly and delivering projects on time.
                \item Won first prize twice (2012 and 2014) in yearly hackathons at Drishti.
                \item Participated in training process at Drishti and delivered training sessions for employees on topics such as Model Driven Architecture, Service Oriented Architecture and Code Generation.
                \item Scored 98.3 percentile in Computer Science Graduate Aptitude Test in Engineering (GATE-2012) among 156000 applicants from all over India.
                \item Revived GNU/Linux Users Group (GLUG-NITH) at NIT Hamirpur and organized introductory Linux workshops for first-year students.
                \item Served as coordinator of Computer Science and Engineering Department in Nimbus'11 (annual technical festival of NIT Hamirpur) and was responsible for organizing technical events.
%                \item Received first prize in coding competitions Dirty Code and Code Crypt during Nimbus’12
                \item Achieved state rank 72 (Himachal Pradesh) in All India Engineering Entrance Exam (AIEEE-2008).

            \end{itemize}
        \end{cvsubsection}
    \end{cvsection}
    
    \begin{cvsection}{Languages}
        \begin{cvsubsection}{}{}{}    
            \begin{itemize}
                \item English (Full Professional Proficiency), German (A1 Level), Hindi (Native).
            \end{itemize}
        \end{cvsubsection}
    \end{cvsection}
    
        \begin{cvsection}{References}

         \printbibliography[heading=none]

    \end{cvsection}
    
\end{document}
