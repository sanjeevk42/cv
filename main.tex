%% The MIT License (MIT)
%%
%% Copyright (c) 2015 Daniil Belyakov
%%
%% Permission is hereby granted, free of charge, to any person obtaining a copy
%% of this software and associated documentation files (the "Software"), to deal
%% in the Software without restriction, including without limitation the rights
%% to use, copy, modify, merge, publish, distribute, sublicense, and/or sell
%% copies of the Software, and to permit persons to whom the Software is
%% furnished to do so, subject to the following conditions:
%%
%% The above copyright notice and this permission notice shall be included in all
%% copies or substantial portions of the Software.
%%
%% THE SOFTWARE IS PROVIDED "AS IS", WITHOUT WARRANTY OF ANY KIND, EXPRESS OR
%% IMPLIED, INCLUDING BUT NOT LIMITED TO THE WARRANTIES OF MERCHANTABILITY,
%% FITNESS FOR A PARTICULAR PURPOSE AND NONINFRINGEMENT. IN NO EVENT SHALL THE
%% AUTHORS OR COPYRIGHT HOLDERS BE LIABLE FOR ANY CLAIM, DAMAGES OR OTHER
%% LIABILITY, WHETHER IN AN ACTION OF CONTRACT, TORT OR OTHERWISE, ARISING FROM,
%% OUT OF OR IN CONNECTION WITH THE SOFTWARE OR THE USE OR OTHER DEALINGS IN THE
%% SOFTWARE.

% The font could be set to Windows-specific Calibri by using the 'calibri' option
\documentclass[]{mcdowellcv}
\usepackage{lmodern}
% For mathematical symbols
\usepackage{amsmath}
\usepackage{url}
\usepackage[svgnames]{xcolor}
\usepackage[colorlinks=true, linkcolor=Maroon, urlcolor=Maroon, citecolor=blue]{hyperref}
\usepackage[symbol]{footmisc}


\renewcommand{\thefootnote}{\fnsymbol{footnote}}


\usepackage[backend=biber,sorting=none]{biblatex}
\addbibresource{ref.bib}


% Set applicant's personal data for header
\name{Sanjeev Kumar}
\email{sanjeevksh42@gmail.com}
\github{https://github.com/sanjeevk42}
\linkedin{https://linkedin.com/in/sanjeevk42}


\address{Phone: +49 17631702520 \linebreak Address: Peter-Paul-Althaus Str. 15 \linebreak Munich 80805}
\contacts{}


\begin{document}
	
	% Print the header
	\makeheader
	
	\begin{cvsection}{Education}
		\begin{cvsubsection}{Munich, Germany}{Technical University of Munich}{Oct 2015 -- March 2018}
			\begin{itemize}
				\item M.Sc. in Informatics (Specialization: Computer Vision and Machine Learning)
				\item Inter-disciplinary Project: Cell Detection in Lens-free Microscopy Videos (published in MICCAI'17) \cite{rempfler-2017}
				\item Coursework: Machine Learning; Multiple View Geometry; Deep Learning for Computer Vision; Vision Based Navigation.
			\end{itemize}
		\end{cvsubsection}
		\begin{cvsubsection}{Hamirpur, India}{National Institute of Technology}{July 2008 -- May 2012}
			\begin{itemize}
				\item B.Tech. in Computer Science and Engineering
				\item Coursework: Object Oriented Programming; Data Structure and Algorithms; Theory of Computation; Operating System; Computer Networks.
			\end{itemize}
		\end{cvsubsection}
	\end{cvsection}
	
	
	% work experience
	\begin{cvsection}{Professional Experience}
		
		\begin{cvsubsection}{Machine Learning Engineer}{Lyft Level 5, Self-Driving Division}{Aug 2018 -- Present}   
			\textit{ } \\
			\textbf{Technologies Used}: Python, C++, PyTorch, OpenCV \\
			\textbf{Semantic Map Generator}
			\begin{itemize}
				\item Primary contributor and owner of the component that generates the HD semantic map required for on car perception and planning.
				%The map has detailed information about different road elements and their connectivity e.g. lane boundaries, traffic lights etc.
				\item Implemented different geo spatial algorithms and ML pipelines to speed the map creation with human annotators in the loop.
			\end{itemize}
			
			\textbf{Traffic Light Placement}
			\begin{itemize}
				\item Implemented a 3D placement algorithm for detecting and localizing traffic lights in point clouds.
				\item Integrated the automatic traffic light placement pipeline with Semantic Map Generator and QC tools for validation/correction 
				by human curators.
			\end{itemize}
			
			\textbf{Road Map Element Detection}
			\begin{itemize}
				\item Implemented an annotation pipeline for creating ground truth datasets to extract geometries for different types of road elements (lanes, crosswalks, arrows, etc.).
				\item Led the implementation of semantic segmentation model training and shape extraction pipeline on orthographically projected camera images. 
				\item Mentored an intern to integrate different road element models in QC tools and map generator.
			\end{itemize}
			
			\textbf{LiDAR Point Cloud Annotation}
			\begin{itemize}
				\item Designed and implemented a 3D detector (PointRCNN) and tracker (Kalman filter-based) pipeline to pre-populate object tracks to speed up the annotation of dynamic agents (cars, pedestrians, etc.) in the LiDAR point cloud.
				\item Worked closely with the team in Palo Alto to integrate the tracker pipeline in the UI tool for point cloud annotation.
			\end{itemize}
			
		\end{cvsubsection}
		
		
		\begin{cvsubsection}{Software Engineer (Part-time)}{Terraloupe Gmbh}{Aug 2016 - April 2017}
			\textbf{Technologies Used}: Python, Keras, Shapely
			\begin{itemize}
				\item Experimented with various deep convolutional network architectures (PSPNet, UNet etc.) for semantic segmentation of roof objects (roof, chimney etc.) from aerial images.
				\item Implemented geo spatial algorithms to generate refined shape boundaries for detected objects. The roof area was calculated from the point cloud of given region and roof object boundaries.
			\end{itemize}
		\end{cvsubsection}
		
		%\newpage        
		
		\begin{cvsubsection}{Software Development Engineer}{Amazon, India}{Oct 2014 -- Oct 2015}  
			\textit{Technologies Used}: Java, RDS, DynamoDB, Herd (Workflow-orchestration Engine)
			\begin{itemize}
				\item Optimized the seller data ingestion pipeline by detecting duplicate offers in daily XML feed.
				\item Built a system that automatically sends notifications (SMS, email) to sellers for older offers on junglee.com and implemented the deletion
				workflow that could be triggered via one-click or a missed call.
			\end{itemize}
		\end{cvsubsection}
		
		\begin{cvsubsection}{Software Developer}{Drishti Soft Solutions, India}{June 2012 -- Oct 2014}
			\textbf{Technologies Used}: Java, JAX-RS, Postgres
			\begin{itemize} 
				\item Led the design and implementation of REST API for the core call center functionality (queuing calls, allocating agents etc.). Worked closely with CRM providers for integrating the API.  
				\item Implemented a real-time monitoring system for analyzing the call volume and SLA in a call center. 
			\end{itemize}
		\end{cvsubsection}
	\end{cvsection}
	
	
	
	
	\begin{cvsection}{Technical Skills}
		\begin{cvsubsection}{}{}{}    
			\begin{itemize}
				\item Programming Languages: Python (Advanced), C++ (Intermidiate), Java (Intermidiate).
				\item ML/CV Toolkits: Tensorflow, PyTorch, OpenCV, ROS, SciPy.
				\item Others: GNU/Linux, Docker, Amazon Web Services.
			\end{itemize}
		\end{cvsubsection}
	\end{cvsection}
	
	\begin{cvsection}{References}
		
		\printbibliography[heading=none]
		
	\end{cvsection}
	
\end{document}