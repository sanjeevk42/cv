%% The MIT License (MIT)
%%
%% Copyright (c) 2015 Daniil Belyakov
%%
%% Permission is hereby granted, free of charge, to any person obtaining a copy
%% of this software and associated documentation files (the "Software"), to deal
%% in the Software without restriction, including without limitation the rights
%% to use, copy, modify, merge, publish, distribute, sublicense, and/or sell
%% copies of the Software, and to permit persons to whom the Software is
%% furnished to do so, subject to the following conditions:
%%
%% The above copyright notice and this permission notice shall be included in all
%% copies or substantial portions of the Software.
%%
%% THE SOFTWARE IS PROVIDED "AS IS", WITHOUT WARRANTY OF ANY KIND, EXPRESS OR
%% IMPLIED, INCLUDING BUT NOT LIMITED TO THE WARRANTIES OF MERCHANTABILITY,
%% FITNESS FOR A PARTICULAR PURPOSE AND NONINFRINGEMENT. IN NO EVENT SHALL THE
%% AUTHORS OR COPYRIGHT HOLDERS BE LIABLE FOR ANY CLAIM, DAMAGES OR OTHER
%% LIABILITY, WHETHER IN AN ACTION OF CONTRACT, TORT OR OTHERWISE, ARISING FROM,
%% OUT OF OR IN CONNECTION WITH THE SOFTWARE OR THE USE OR OTHER DEALINGS IN THE
%% SOFTWARE.

% The font could be set to Windows-specific Calibri by using the 'calibri' option
\documentclass[]{mcdowellcv}
\usepackage{lmodern}
% For mathematical symbols
\usepackage{amsmath}
\usepackage{url}
\usepackage{hyperref}

\usepackage[backend=biber]{biblatex}
\addbibresource{ref.bib}


% Set applicant's personal data for header
\name{Sanjeev Kumar}
\address{Felsennelkenanger 13 \linebreak Munich 80937}
\contacts{+49 123456789 \linebreak sanjeev90an@gmail.com}

\begin{document}

	% Print the header
	\makeheader
	
	% academic projects
	\begin{cvsection}{Academic Projects}
	\begin{cvsubsection}[2]{Master Thesis}{CAMP\footnote{\href{http://campar.in.tum.de/Chair/ResearchIssueComputerVision}{Computer Vision Team, Computer Aided Medical Procedures, Technical University of Munich}}, Technical University of Munich}{June 2017 -- Present}
    Multiple Action Prediction in Reinforcement Learning (Python, Tensorflow, OpenAI Gym) \cite{anonymous2018predicting}
        \begin{itemize}
	        \item Proposed a new formulation for policy gradient reinforcement learning algorithms in continuous action space problems.
	        \item The method predicts multiple action values at each state which facilitates better exploration during training and the agent converges to a better policy.
	        \item Evaluated and compared the performance of the proposed formulation against other continuous control algorithms (A3C, DDPG) on various Mujoco environments.
	    \end{itemize}
	\end{cvsubsection}
	
	\begin{cvsubsection}{Inter-Disciplinary Project}{Technical University of Munich}{Oct 2016 -- Mar 2017}
	Cell Detection in Lens-free Microscopy Videos (Python, Keras) \cite{rempfler-2017}
    	\begin{itemize}
    	\item This aim of the project was to detect and localize cells in lens free microscopy image sequences using deep convolutional neural networks (CNN).
        \item In this we experimented with different deep CNN architectures (FCN, UNet, DetectNet) and achieved best detection score of 95\% (F1) with fully convolutional ResNet-50.
    	\end{itemize}
	\end{cvsubsection}
	
	\end{cvsection}
	
	% work experience
	\begin{cvsection}{Employment}
		\begin{cvsubsection}{Machine Learning Engineer}{Logivations Gmbh}{April 2017 - Present}
			Identification Workplace			
			\begin{itemize}
				\item Applied various object detection algorithms (RFCN, Faster RCNN, Yolo etc.) for detecting different warehouse objects.
				\item Designed and implemented framework to integrate and deploy new models easily across different platforms.
			\end{itemize}
		\end{cvsubsection}
		
		\begin{cvsubsection}{Student Tutor}{Technical University of Munich}{Oct 2017 -- Present}	
		Tracking and Detection in Computer Vision (IN2210) \footnote{\href{http://campar.in.tum.de/Chair/TeachingWs17TDCV}{Tracking and Detection in Computer Vision}}
			\begin{itemize}
				\item The course is regularly offered to master students at TU Munich and is attended by more than 100 students. 
				\item Involved in creating assignments for the course and helped students with the homework. 
			\end{itemize}
		\end{cvsubsection}
		
		\begin{cvsubsection}{Machine Learning Engineer}{Terraloupe Gmbh}{Aug 2017 - April 2017}
			Roof Object Segmentation (Python, Keras, Shapely)
			\begin{itemize}
				\item Experimented with various deep convolutional network architectures (PSPNet, UNet etc.) for semantic segmentation of roof objects (roof, chimney etc.) from aerial images.
				\item Implemented various computational geometric algorithms (polygon merge, line merge etc.) to generate refined shape boundaries for detected objects.
			\end{itemize}
		\end{cvsubsection}
		
		\begin{cvsubsection}{Software Development Engineer}{Amazon, India}{Oct 2014 -- Oct 2015}		
		Missed Call Based Expiry (Java, SQS, DynamoDB)
			\begin{itemize}
				\item Developed an application which sends notifications(SMS, email) to customers for older ads live on junglee.com and customers can then delete their ads by giving a missed call on a toll free number. 
                \item Using this mechanism, the Ad Defect Rate on junglee.com went down by 15\%.
			\end{itemize}
		\end{cvsubsection}
		
		\begin{cvsubsection}{Senior Software Developer}{Drishti Soft Solutions, India}{June 2012 -- Sept 2014}
			REST API for Ameyo (Java, Xtend, Jersey)
			\begin{itemize}
				\item Designed REST based APIs for Ameyo (Drishti's call center software suite). The APIs were modeled in ecore domain model and wrote a code generator plugin in eclipse for generating Java code for APIs. 
				\item This API provided the framework to develop call center applications easily.
			\end{itemize}
			
			Stats Manager (Java, H2)
			\begin{itemize}
				\item Designed and implemented a module for generating real time call center statistics. 
				\item The module listened to call center application events and maintained in-memory database of stats which could be consumed by API calls or by subscribing to notifications.
			\end{itemize}
		\end{cvsubsection}
	\end{cvsection}
	
  \newpage
	\begin{cvsection}{Education}
		\begin{cvsubsection}{Munich, Germany}{Technical University of Munich}{Oct 2015 -- Mar 2018}
			\begin{itemize}
				\item M.Sc. in Informatics, GPA: 1.9/1.0
				\item Graduate Coursework: Machine Learning; Variational Methods; Deep Learning for Computer Vision; Vision Based Navigation; Programming Languages; Tracking and Detection
			\end{itemize}
		\end{cvsubsection}
		\begin{cvsubsection}{Hamirpur, India}{National Institute of Technology}{July 2008 -- May 2012}
			\begin{itemize}
				\item B.Tech. in Computer Science and Engineering,  CGPA: 7.69/10
				\item Undergraduate Coursework: Object Oriented Programming; Data Structure and Algorithms; Theory of Computation; Operating System; Computer Networks 
			\end{itemize}
		\end{cvsubsection}
	\end{cvsection}
	
	\begin{cvsection}{Additional Experience and Awards}
		\begin{cvsubsection}{}{}{}	
			\begin{itemize}
				\item \textbf{Third Prize, Junglee Hackathon:} Participated in annual hackathon at Amazon and won third prize in Hackathon’15.
				\item \textbf{Excellence Award:} Awarded with Drishti Excellence Award in October 2013, for gaining domain knowledge quickly and delivering projects on time.
				\item Mentored interns and took training sessions of new employees at Drishti and Amazon.
			\end{itemize}
		\end{cvsubsection}
	\end{cvsection}
	
	\begin{cvsection}{Languages and Technologies}
		\begin{cvsubsection}{}{}{}	
			\begin{itemize}
				\item Python (Advanced); Java (Advanced); C++ (Intermediate); C (Intermediate); Matlab (Basic)
				\item Tensorflow; OpenCV; ROS; REST
			\end{itemize}
		\end{cvsubsection}
	\end{cvsection}
	
		\begin{cvsection}{References}

     	\printbibliography[heading=none]

	\end{cvsection}
	
\end{document}
